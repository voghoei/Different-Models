
% Default to the notebook output style

    


% Inherit from the specified cell style.




    
\documentclass[11pt]{article}

    
    
    \usepackage[T1]{fontenc}
    % Nicer default font (+ math font) than Computer Modern for most use cases
    \usepackage{mathpazo}

    % Basic figure setup, for now with no caption control since it's done
    % automatically by Pandoc (which extracts ![](path) syntax from Markdown).
    \usepackage{graphicx}
    % We will generate all images so they have a width \maxwidth. This means
    % that they will get their normal width if they fit onto the page, but
    % are scaled down if they would overflow the margins.
    \makeatletter
    \def\maxwidth{\ifdim\Gin@nat@width>\linewidth\linewidth
    \else\Gin@nat@width\fi}
    \makeatother
    \let\Oldincludegraphics\includegraphics
    % Set max figure width to be 80% of text width, for now hardcoded.
    \renewcommand{\includegraphics}[1]{\Oldincludegraphics[width=.8\maxwidth]{#1}}
    % Ensure that by default, figures have no caption (until we provide a
    % proper Figure object with a Caption API and a way to capture that
    % in the conversion process - todo).
    \usepackage{caption}
    \DeclareCaptionLabelFormat{nolabel}{}
    \captionsetup{labelformat=nolabel}

    \usepackage{adjustbox} % Used to constrain images to a maximum size 
    \usepackage{xcolor} % Allow colors to be defined
    \usepackage{enumerate} % Needed for markdown enumerations to work
    \usepackage{geometry} % Used to adjust the document margins
    \usepackage{amsmath} % Equations
    \usepackage{amssymb} % Equations
    \usepackage{textcomp} % defines textquotesingle
    % Hack from http://tex.stackexchange.com/a/47451/13684:
    \AtBeginDocument{%
        \def\PYZsq{\textquotesingle}% Upright quotes in Pygmentized code
    }
    \usepackage{upquote} % Upright quotes for verbatim code
    \usepackage{eurosym} % defines \euro
    \usepackage[mathletters]{ucs} % Extended unicode (utf-8) support
    \usepackage[utf8x]{inputenc} % Allow utf-8 characters in the tex document
    \usepackage{fancyvrb} % verbatim replacement that allows latex
    \usepackage{grffile} % extends the file name processing of package graphics 
                         % to support a larger range 
    % The hyperref package gives us a pdf with properly built
    % internal navigation ('pdf bookmarks' for the table of contents,
    % internal cross-reference links, web links for URLs, etc.)
    \usepackage{hyperref}
    \usepackage{longtable} % longtable support required by pandoc >1.10
    \usepackage{booktabs}  % table support for pandoc > 1.12.2
    \usepackage[inline]{enumitem} % IRkernel/repr support (it uses the enumerate* environment)
    \usepackage[normalem]{ulem} % ulem is needed to support strikethroughs (\sout)
                                % normalem makes italics be italics, not underlines
    

    
    
    % Colors for the hyperref package
    \definecolor{urlcolor}{rgb}{0,.145,.698}
    \definecolor{linkcolor}{rgb}{.71,0.21,0.01}
    \definecolor{citecolor}{rgb}{.12,.54,.11}

    % ANSI colors
    \definecolor{ansi-black}{HTML}{3E424D}
    \definecolor{ansi-black-intense}{HTML}{282C36}
    \definecolor{ansi-red}{HTML}{E75C58}
    \definecolor{ansi-red-intense}{HTML}{B22B31}
    \definecolor{ansi-green}{HTML}{00A250}
    \definecolor{ansi-green-intense}{HTML}{007427}
    \definecolor{ansi-yellow}{HTML}{DDB62B}
    \definecolor{ansi-yellow-intense}{HTML}{B27D12}
    \definecolor{ansi-blue}{HTML}{208FFB}
    \definecolor{ansi-blue-intense}{HTML}{0065CA}
    \definecolor{ansi-magenta}{HTML}{D160C4}
    \definecolor{ansi-magenta-intense}{HTML}{A03196}
    \definecolor{ansi-cyan}{HTML}{60C6C8}
    \definecolor{ansi-cyan-intense}{HTML}{258F8F}
    \definecolor{ansi-white}{HTML}{C5C1B4}
    \definecolor{ansi-white-intense}{HTML}{A1A6B2}

    % commands and environments needed by pandoc snippets
    % extracted from the output of `pandoc -s`
    \providecommand{\tightlist}{%
      \setlength{\itemsep}{0pt}\setlength{\parskip}{0pt}}
    \DefineVerbatimEnvironment{Highlighting}{Verbatim}{commandchars=\\\{\}}
    % Add ',fontsize=\small' for more characters per line
    \newenvironment{Shaded}{}{}
    \newcommand{\KeywordTok}[1]{\textcolor[rgb]{0.00,0.44,0.13}{\textbf{{#1}}}}
    \newcommand{\DataTypeTok}[1]{\textcolor[rgb]{0.56,0.13,0.00}{{#1}}}
    \newcommand{\DecValTok}[1]{\textcolor[rgb]{0.25,0.63,0.44}{{#1}}}
    \newcommand{\BaseNTok}[1]{\textcolor[rgb]{0.25,0.63,0.44}{{#1}}}
    \newcommand{\FloatTok}[1]{\textcolor[rgb]{0.25,0.63,0.44}{{#1}}}
    \newcommand{\CharTok}[1]{\textcolor[rgb]{0.25,0.44,0.63}{{#1}}}
    \newcommand{\StringTok}[1]{\textcolor[rgb]{0.25,0.44,0.63}{{#1}}}
    \newcommand{\CommentTok}[1]{\textcolor[rgb]{0.38,0.63,0.69}{\textit{{#1}}}}
    \newcommand{\OtherTok}[1]{\textcolor[rgb]{0.00,0.44,0.13}{{#1}}}
    \newcommand{\AlertTok}[1]{\textcolor[rgb]{1.00,0.00,0.00}{\textbf{{#1}}}}
    \newcommand{\FunctionTok}[1]{\textcolor[rgb]{0.02,0.16,0.49}{{#1}}}
    \newcommand{\RegionMarkerTok}[1]{{#1}}
    \newcommand{\ErrorTok}[1]{\textcolor[rgb]{1.00,0.00,0.00}{\textbf{{#1}}}}
    \newcommand{\NormalTok}[1]{{#1}}
    
    % Additional commands for more recent versions of Pandoc
    \newcommand{\ConstantTok}[1]{\textcolor[rgb]{0.53,0.00,0.00}{{#1}}}
    \newcommand{\SpecialCharTok}[1]{\textcolor[rgb]{0.25,0.44,0.63}{{#1}}}
    \newcommand{\VerbatimStringTok}[1]{\textcolor[rgb]{0.25,0.44,0.63}{{#1}}}
    \newcommand{\SpecialStringTok}[1]{\textcolor[rgb]{0.73,0.40,0.53}{{#1}}}
    \newcommand{\ImportTok}[1]{{#1}}
    \newcommand{\DocumentationTok}[1]{\textcolor[rgb]{0.73,0.13,0.13}{\textit{{#1}}}}
    \newcommand{\AnnotationTok}[1]{\textcolor[rgb]{0.38,0.63,0.69}{\textbf{\textit{{#1}}}}}
    \newcommand{\CommentVarTok}[1]{\textcolor[rgb]{0.38,0.63,0.69}{\textbf{\textit{{#1}}}}}
    \newcommand{\VariableTok}[1]{\textcolor[rgb]{0.10,0.09,0.49}{{#1}}}
    \newcommand{\ControlFlowTok}[1]{\textcolor[rgb]{0.00,0.44,0.13}{\textbf{{#1}}}}
    \newcommand{\OperatorTok}[1]{\textcolor[rgb]{0.40,0.40,0.40}{{#1}}}
    \newcommand{\BuiltInTok}[1]{{#1}}
    \newcommand{\ExtensionTok}[1]{{#1}}
    \newcommand{\PreprocessorTok}[1]{\textcolor[rgb]{0.74,0.48,0.00}{{#1}}}
    \newcommand{\AttributeTok}[1]{\textcolor[rgb]{0.49,0.56,0.16}{{#1}}}
    \newcommand{\InformationTok}[1]{\textcolor[rgb]{0.38,0.63,0.69}{\textbf{\textit{{#1}}}}}
    \newcommand{\WarningTok}[1]{\textcolor[rgb]{0.38,0.63,0.69}{\textbf{\textit{{#1}}}}}
    
    
    % Define a nice break command that doesn't care if a line doesn't already
    % exist.
    \def\br{\hspace*{\fill} \\* }
    % Math Jax compatability definitions
    \def\gt{>}
    \def\lt{<}
    % Document parameters
    \title{Recursion}
    
    
    

    % Pygments definitions
    
\makeatletter
\def\PY@reset{\let\PY@it=\relax \let\PY@bf=\relax%
    \let\PY@ul=\relax \let\PY@tc=\relax%
    \let\PY@bc=\relax \let\PY@ff=\relax}
\def\PY@tok#1{\csname PY@tok@#1\endcsname}
\def\PY@toks#1+{\ifx\relax#1\empty\else%
    \PY@tok{#1}\expandafter\PY@toks\fi}
\def\PY@do#1{\PY@bc{\PY@tc{\PY@ul{%
    \PY@it{\PY@bf{\PY@ff{#1}}}}}}}
\def\PY#1#2{\PY@reset\PY@toks#1+\relax+\PY@do{#2}}

\expandafter\def\csname PY@tok@w\endcsname{\def\PY@tc##1{\textcolor[rgb]{0.73,0.73,0.73}{##1}}}
\expandafter\def\csname PY@tok@c\endcsname{\let\PY@it=\textit\def\PY@tc##1{\textcolor[rgb]{0.25,0.50,0.50}{##1}}}
\expandafter\def\csname PY@tok@cp\endcsname{\def\PY@tc##1{\textcolor[rgb]{0.74,0.48,0.00}{##1}}}
\expandafter\def\csname PY@tok@k\endcsname{\let\PY@bf=\textbf\def\PY@tc##1{\textcolor[rgb]{0.00,0.50,0.00}{##1}}}
\expandafter\def\csname PY@tok@kp\endcsname{\def\PY@tc##1{\textcolor[rgb]{0.00,0.50,0.00}{##1}}}
\expandafter\def\csname PY@tok@kt\endcsname{\def\PY@tc##1{\textcolor[rgb]{0.69,0.00,0.25}{##1}}}
\expandafter\def\csname PY@tok@o\endcsname{\def\PY@tc##1{\textcolor[rgb]{0.40,0.40,0.40}{##1}}}
\expandafter\def\csname PY@tok@ow\endcsname{\let\PY@bf=\textbf\def\PY@tc##1{\textcolor[rgb]{0.67,0.13,1.00}{##1}}}
\expandafter\def\csname PY@tok@nb\endcsname{\def\PY@tc##1{\textcolor[rgb]{0.00,0.50,0.00}{##1}}}
\expandafter\def\csname PY@tok@nf\endcsname{\def\PY@tc##1{\textcolor[rgb]{0.00,0.00,1.00}{##1}}}
\expandafter\def\csname PY@tok@nc\endcsname{\let\PY@bf=\textbf\def\PY@tc##1{\textcolor[rgb]{0.00,0.00,1.00}{##1}}}
\expandafter\def\csname PY@tok@nn\endcsname{\let\PY@bf=\textbf\def\PY@tc##1{\textcolor[rgb]{0.00,0.00,1.00}{##1}}}
\expandafter\def\csname PY@tok@ne\endcsname{\let\PY@bf=\textbf\def\PY@tc##1{\textcolor[rgb]{0.82,0.25,0.23}{##1}}}
\expandafter\def\csname PY@tok@nv\endcsname{\def\PY@tc##1{\textcolor[rgb]{0.10,0.09,0.49}{##1}}}
\expandafter\def\csname PY@tok@no\endcsname{\def\PY@tc##1{\textcolor[rgb]{0.53,0.00,0.00}{##1}}}
\expandafter\def\csname PY@tok@nl\endcsname{\def\PY@tc##1{\textcolor[rgb]{0.63,0.63,0.00}{##1}}}
\expandafter\def\csname PY@tok@ni\endcsname{\let\PY@bf=\textbf\def\PY@tc##1{\textcolor[rgb]{0.60,0.60,0.60}{##1}}}
\expandafter\def\csname PY@tok@na\endcsname{\def\PY@tc##1{\textcolor[rgb]{0.49,0.56,0.16}{##1}}}
\expandafter\def\csname PY@tok@nt\endcsname{\let\PY@bf=\textbf\def\PY@tc##1{\textcolor[rgb]{0.00,0.50,0.00}{##1}}}
\expandafter\def\csname PY@tok@nd\endcsname{\def\PY@tc##1{\textcolor[rgb]{0.67,0.13,1.00}{##1}}}
\expandafter\def\csname PY@tok@s\endcsname{\def\PY@tc##1{\textcolor[rgb]{0.73,0.13,0.13}{##1}}}
\expandafter\def\csname PY@tok@sd\endcsname{\let\PY@it=\textit\def\PY@tc##1{\textcolor[rgb]{0.73,0.13,0.13}{##1}}}
\expandafter\def\csname PY@tok@si\endcsname{\let\PY@bf=\textbf\def\PY@tc##1{\textcolor[rgb]{0.73,0.40,0.53}{##1}}}
\expandafter\def\csname PY@tok@se\endcsname{\let\PY@bf=\textbf\def\PY@tc##1{\textcolor[rgb]{0.73,0.40,0.13}{##1}}}
\expandafter\def\csname PY@tok@sr\endcsname{\def\PY@tc##1{\textcolor[rgb]{0.73,0.40,0.53}{##1}}}
\expandafter\def\csname PY@tok@ss\endcsname{\def\PY@tc##1{\textcolor[rgb]{0.10,0.09,0.49}{##1}}}
\expandafter\def\csname PY@tok@sx\endcsname{\def\PY@tc##1{\textcolor[rgb]{0.00,0.50,0.00}{##1}}}
\expandafter\def\csname PY@tok@m\endcsname{\def\PY@tc##1{\textcolor[rgb]{0.40,0.40,0.40}{##1}}}
\expandafter\def\csname PY@tok@gh\endcsname{\let\PY@bf=\textbf\def\PY@tc##1{\textcolor[rgb]{0.00,0.00,0.50}{##1}}}
\expandafter\def\csname PY@tok@gu\endcsname{\let\PY@bf=\textbf\def\PY@tc##1{\textcolor[rgb]{0.50,0.00,0.50}{##1}}}
\expandafter\def\csname PY@tok@gd\endcsname{\def\PY@tc##1{\textcolor[rgb]{0.63,0.00,0.00}{##1}}}
\expandafter\def\csname PY@tok@gi\endcsname{\def\PY@tc##1{\textcolor[rgb]{0.00,0.63,0.00}{##1}}}
\expandafter\def\csname PY@tok@gr\endcsname{\def\PY@tc##1{\textcolor[rgb]{1.00,0.00,0.00}{##1}}}
\expandafter\def\csname PY@tok@ge\endcsname{\let\PY@it=\textit}
\expandafter\def\csname PY@tok@gs\endcsname{\let\PY@bf=\textbf}
\expandafter\def\csname PY@tok@gp\endcsname{\let\PY@bf=\textbf\def\PY@tc##1{\textcolor[rgb]{0.00,0.00,0.50}{##1}}}
\expandafter\def\csname PY@tok@go\endcsname{\def\PY@tc##1{\textcolor[rgb]{0.53,0.53,0.53}{##1}}}
\expandafter\def\csname PY@tok@gt\endcsname{\def\PY@tc##1{\textcolor[rgb]{0.00,0.27,0.87}{##1}}}
\expandafter\def\csname PY@tok@err\endcsname{\def\PY@bc##1{\setlength{\fboxsep}{0pt}\fcolorbox[rgb]{1.00,0.00,0.00}{1,1,1}{\strut ##1}}}
\expandafter\def\csname PY@tok@kc\endcsname{\let\PY@bf=\textbf\def\PY@tc##1{\textcolor[rgb]{0.00,0.50,0.00}{##1}}}
\expandafter\def\csname PY@tok@kd\endcsname{\let\PY@bf=\textbf\def\PY@tc##1{\textcolor[rgb]{0.00,0.50,0.00}{##1}}}
\expandafter\def\csname PY@tok@kn\endcsname{\let\PY@bf=\textbf\def\PY@tc##1{\textcolor[rgb]{0.00,0.50,0.00}{##1}}}
\expandafter\def\csname PY@tok@kr\endcsname{\let\PY@bf=\textbf\def\PY@tc##1{\textcolor[rgb]{0.00,0.50,0.00}{##1}}}
\expandafter\def\csname PY@tok@bp\endcsname{\def\PY@tc##1{\textcolor[rgb]{0.00,0.50,0.00}{##1}}}
\expandafter\def\csname PY@tok@fm\endcsname{\def\PY@tc##1{\textcolor[rgb]{0.00,0.00,1.00}{##1}}}
\expandafter\def\csname PY@tok@vc\endcsname{\def\PY@tc##1{\textcolor[rgb]{0.10,0.09,0.49}{##1}}}
\expandafter\def\csname PY@tok@vg\endcsname{\def\PY@tc##1{\textcolor[rgb]{0.10,0.09,0.49}{##1}}}
\expandafter\def\csname PY@tok@vi\endcsname{\def\PY@tc##1{\textcolor[rgb]{0.10,0.09,0.49}{##1}}}
\expandafter\def\csname PY@tok@vm\endcsname{\def\PY@tc##1{\textcolor[rgb]{0.10,0.09,0.49}{##1}}}
\expandafter\def\csname PY@tok@sa\endcsname{\def\PY@tc##1{\textcolor[rgb]{0.73,0.13,0.13}{##1}}}
\expandafter\def\csname PY@tok@sb\endcsname{\def\PY@tc##1{\textcolor[rgb]{0.73,0.13,0.13}{##1}}}
\expandafter\def\csname PY@tok@sc\endcsname{\def\PY@tc##1{\textcolor[rgb]{0.73,0.13,0.13}{##1}}}
\expandafter\def\csname PY@tok@dl\endcsname{\def\PY@tc##1{\textcolor[rgb]{0.73,0.13,0.13}{##1}}}
\expandafter\def\csname PY@tok@s2\endcsname{\def\PY@tc##1{\textcolor[rgb]{0.73,0.13,0.13}{##1}}}
\expandafter\def\csname PY@tok@sh\endcsname{\def\PY@tc##1{\textcolor[rgb]{0.73,0.13,0.13}{##1}}}
\expandafter\def\csname PY@tok@s1\endcsname{\def\PY@tc##1{\textcolor[rgb]{0.73,0.13,0.13}{##1}}}
\expandafter\def\csname PY@tok@mb\endcsname{\def\PY@tc##1{\textcolor[rgb]{0.40,0.40,0.40}{##1}}}
\expandafter\def\csname PY@tok@mf\endcsname{\def\PY@tc##1{\textcolor[rgb]{0.40,0.40,0.40}{##1}}}
\expandafter\def\csname PY@tok@mh\endcsname{\def\PY@tc##1{\textcolor[rgb]{0.40,0.40,0.40}{##1}}}
\expandafter\def\csname PY@tok@mi\endcsname{\def\PY@tc##1{\textcolor[rgb]{0.40,0.40,0.40}{##1}}}
\expandafter\def\csname PY@tok@il\endcsname{\def\PY@tc##1{\textcolor[rgb]{0.40,0.40,0.40}{##1}}}
\expandafter\def\csname PY@tok@mo\endcsname{\def\PY@tc##1{\textcolor[rgb]{0.40,0.40,0.40}{##1}}}
\expandafter\def\csname PY@tok@ch\endcsname{\let\PY@it=\textit\def\PY@tc##1{\textcolor[rgb]{0.25,0.50,0.50}{##1}}}
\expandafter\def\csname PY@tok@cm\endcsname{\let\PY@it=\textit\def\PY@tc##1{\textcolor[rgb]{0.25,0.50,0.50}{##1}}}
\expandafter\def\csname PY@tok@cpf\endcsname{\let\PY@it=\textit\def\PY@tc##1{\textcolor[rgb]{0.25,0.50,0.50}{##1}}}
\expandafter\def\csname PY@tok@c1\endcsname{\let\PY@it=\textit\def\PY@tc##1{\textcolor[rgb]{0.25,0.50,0.50}{##1}}}
\expandafter\def\csname PY@tok@cs\endcsname{\let\PY@it=\textit\def\PY@tc##1{\textcolor[rgb]{0.25,0.50,0.50}{##1}}}

\def\PYZbs{\char`\\}
\def\PYZus{\char`\_}
\def\PYZob{\char`\{}
\def\PYZcb{\char`\}}
\def\PYZca{\char`\^}
\def\PYZam{\char`\&}
\def\PYZlt{\char`\<}
\def\PYZgt{\char`\>}
\def\PYZsh{\char`\#}
\def\PYZpc{\char`\%}
\def\PYZdl{\char`\$}
\def\PYZhy{\char`\-}
\def\PYZsq{\char`\'}
\def\PYZdq{\char`\"}
\def\PYZti{\char`\~}
% for compatibility with earlier versions
\def\PYZat{@}
\def\PYZlb{[}
\def\PYZrb{]}
\makeatother


    % Exact colors from NB
    \definecolor{incolor}{rgb}{0.0, 0.0, 0.5}
    \definecolor{outcolor}{rgb}{0.545, 0.0, 0.0}



    
    % Prevent overflowing lines due to hard-to-break entities
    \sloppy 
    % Setup hyperref package
    \hypersetup{
      breaklinks=true,  % so long urls are correctly broken across lines
      colorlinks=true,
      urlcolor=urlcolor,
      linkcolor=linkcolor,
      citecolor=citecolor,
      }
    % Slightly bigger margins than the latex defaults
    
    \geometry{verbose,tmargin=1in,bmargin=1in,lmargin=1in,rmargin=1in}
    
    

    \begin{document}
    
    
    \maketitle
    
    

    
    Recursion in Java

Navid Hashemi Tonekaboni

May 31, 2019

 

    What is Recursion?

    In order to understand recursion, you must first understand recursion.

    

    What is Recursion?

    

    Defintion of Recursion

    

    To prevent an infinite loop from occurring, we have what is called a
base case.

     Base Case is the most `basic' form of the problem. The solution to the
base case should be simple and easy to calculate (without any more
recursion).

    Recursive Definition

    To create a recursive definition of some concept, we need to establish
two things:

    \begin{itemize}
\tightlist
\item
  Base Case: create a non-recursive definition as a "base".
\item
  Recursive Case: create a definition in terms of itself, changing it
  somehow (usually towards the base case).
\end{itemize}

    Let's Solve a Problem

\begin{itemize}
\tightlist
\item
  The CEO of a company asks top executives to collect feedback reports
  from everyone in the company.
\end{itemize}

    \begin{itemize}
\tightlist
\item
  Iterative Solution: The CEO asks each employees to provide him/her
  with their feedbacks
\end{itemize}

    \begin{itemize}
\tightlist
\item
  Recursive Solution:

  \begin{itemize}
  \tightlist
  \item
    Base Case: If you don't have anyone under your supervision, pass
    your feedback report on to your manager.
  \item
    Recursive Case: If you have someone under your supervision, collect
    the feedback reports from all your subordinates, add your own
    report, and pass them on to your manager.
  \end{itemize}
\end{itemize}

    

    

    

    

    

    Another Recursive Function Example: Factorial

    A Factorial is the product of all positive integers less than or equal
to n. The Factorial of n would be n! Example: 7! = 7 X 6 X 5 X 4 X 3 X 2
X 1 

    \begin{itemize}
\tightlist
\item
  Base Case: ~ ~ n==0 ~ ~-\/-\/-\textgreater{}~ ~ result is 1
\item
  Recursive Case: ~ ~ n\textgreater{}0 ~ ~-\/-\/-\textgreater{}~ ~
  result is (n-1)! * n
\end{itemize}

    Factorial

    Factorial: Recursive Approach

    \begin{Verbatim}[commandchars=\\\{\}]
{\color{incolor}In [{\color{incolor}5}]:} \PY{k+kt}{int} \PY{n+nf}{factorial}\PY{o}{(}\PY{k+kt}{int} \PY{n}{n}\PY{o}{)}\PY{o}{\PYZob{}}   
            \PY{k}{if} \PY{o}{(}\PY{n}{n} \PY{o}{=}\PY{o}{=} \PY{l+m+mi}{0}\PY{o}{)}    
                \PY{k}{return} \PY{l+m+mi}{1}\PY{o}{;}    
            \PY{k}{else}    
                \PY{k}{return}\PY{o}{(}\PY{n}{n} \PY{o}{*} \PY{n}{factorial}\PY{o}{(}\PY{n}{n}\PY{o}{\PYZhy{}}\PY{l+m+mi}{1}\PY{o}{)}\PY{o}{)}\PY{o}{;}    
        \PY{o}{\PYZcb{}}
\end{Verbatim}


    \begin{Verbatim}[commandchars=\\\{\}]
{\color{incolor}In [{\color{incolor}2}]:} \PY{k+kt}{int} \PY{n}{n} \PY{o}{=} \PY{l+m+mi}{5}\PY{o}{;}
        \PY{k+kt}{int} \PY{n}{fact} \PY{o}{=} \PY{n}{factorial}\PY{o}{(}\PY{n}{n}\PY{o}{)}\PY{o}{;}   
        \PY{n}{System}\PY{o}{.}\PY{n+na}{out}\PY{o}{.}\PY{n+na}{println}\PY{o}{(}\PY{l+s}{\PYZdq{}Factorial of \PYZdq{}}\PY{o}{+} \PY{n}{n} \PY{o}{+}\PY{l+s}{\PYZdq{} is: \PYZdq{}}\PY{o}{+} \PY{n}{fact}\PY{o}{)}\PY{o}{;}
\end{Verbatim}


    \begin{Verbatim}[commandchars=\\\{\}]
Factorial of 5 is: 120

    \end{Verbatim}

    Factorial: Iterative Approach

    \begin{Verbatim}[commandchars=\\\{\}]
{\color{incolor}In [{\color{incolor}3}]:} \PY{k+kt}{int} \PY{n}{fact} \PY{o}{=} \PY{l+m+mi}{1}\PY{o}{;}  
        \PY{k+kt}{int} \PY{n}{n} \PY{o}{=} \PY{l+m+mi}{5}\PY{o}{;}      
        \PY{k}{for}\PY{o}{(}\PY{k+kt}{int} \PY{n}{i}\PY{o}{=}\PY{l+m+mi}{1}\PY{o}{;} \PY{n}{i} \PY{o}{\PYZlt{}}\PY{o}{=} \PY{n}{n}\PY{o}{;} \PY{n}{i}\PY{o}{+}\PY{o}{+}\PY{o}{)}  
            \PY{n}{fact} \PY{o}{=} \PY{n}{fact}\PY{o}{*}\PY{n}{i}\PY{o}{;}   
        \PY{n}{System}\PY{o}{.}\PY{n+na}{out}\PY{o}{.}\PY{n+na}{println}\PY{o}{(}\PY{l+s}{\PYZdq{}Factorial of \PYZdq{}}\PY{o}{+} \PY{n}{n} \PY{o}{+}\PY{l+s}{\PYZdq{} is: \PYZdq{}}\PY{o}{+}\PY{n}{fact}\PY{o}{)}\PY{o}{;}
\end{Verbatim}


    \begin{Verbatim}[commandchars=\\\{\}]
Factorial of 5 is: 120

    \end{Verbatim}

    Another Recursive Function Example: Fibonacci Numbers

    Fibonacci Numbers

    Fibonacci Numbers

    Base Cases: * fibonacci (0) return 1 ~ ~ \& * fibonacci (1) return 1 

    Recusrive Case: * fibonacci(n) = fibonacci(n-1) + fibonacci(n-2)

    Fibonacci: Recursive Approach

    \begin{Verbatim}[commandchars=\\\{\}]
{\color{incolor}In [{\color{incolor} }]:} \PY{k+kt}{int} \PY{n+nf}{fib}\PY{o}{(}\PY{k+kt}{int} \PY{n}{n}\PY{o}{)} \PY{o}{\PYZob{}}
            \PY{n}{System}\PY{o}{.}\PY{n+na}{out}\PY{o}{.}\PY{n+na}{print}\PY{o}{(}\PY{n}{n}\PY{o}{+}\PY{l+s}{\PYZdq{} \PYZdq{}}\PY{o}{)}\PY{o}{;} 
            \PY{k}{if} \PY{o}{(}\PY{n}{n} \PY{o}{=}\PY{o}{=} \PY{l+m+mi}{1}\PY{o}{)} \PY{k}{return} \PY{l+m+mi}{1}\PY{o}{;}
            \PY{k}{if} \PY{o}{(}\PY{n}{n} \PY{o}{=}\PY{o}{=} \PY{l+m+mi}{2}\PY{o}{)} \PY{k}{return} \PY{l+m+mi}{1}\PY{o}{;}
            \PY{k}{return} \PY{n}{fib}\PY{o}{(}\PY{n}{n}\PY{o}{\PYZhy{}}\PY{l+m+mi}{2}\PY{o}{)} \PY{o}{+} \PY{n}{fib}\PY{o}{(}\PY{n}{n}\PY{o}{\PYZhy{}}\PY{l+m+mi}{1}\PY{o}{)}\PY{o}{;}\PY{o}{\PYZcb{}}
\end{Verbatim}


    \begin{Verbatim}[commandchars=\\\{\}]
{\color{incolor}In [{\color{incolor} }]:} \PY{k+kt}{int} \PY{n}{number} \PY{o}{=} \PY{l+m+mi}{5}\PY{o}{;}
        \PY{n}{System}\PY{o}{.}\PY{n+na}{out}\PY{o}{.}\PY{n+na}{println}\PY{o}{(}\PY{l+s}{\PYZdq{}Steps are :\PYZdq{}}\PY{o}{)}\PY{o}{;}
        \PY{n}{System}\PY{o}{.}\PY{n+na}{out}\PY{o}{.}\PY{n+na}{println}\PY{o}{(}\PY{l+s}{\PYZdq{}\PYZbs{}nThe \PYZdq{}}\PY{o}{+} \PY{n}{number} \PY{o}{+}\PY{l+s}{\PYZdq{} th value is: \PYZdq{}} \PY{o}{+} \PY{n}{fib}\PY{o}{(}\PY{n}{number}\PY{o}{)}\PY{o}{)}\PY{o}{;}
\end{Verbatim}


    

    Fibonacci: Recursive Approach

    \begin{Verbatim}[commandchars=\\\{\}]
{\color{incolor}In [{\color{incolor} }]:} \PY{k+kt}{int} \PY{n+nf}{fib}\PY{o}{(}\PY{k+kt}{int} \PY{n}{n}\PY{o}{)} \PY{o}{\PYZob{}}
            \PY{k}{if} \PY{o}{(}\PY{n}{n} \PY{o}{=}\PY{o}{=} \PY{l+m+mi}{1}\PY{o}{)} \PY{k}{return} \PY{l+m+mi}{1}\PY{o}{;}
            \PY{k}{if} \PY{o}{(}\PY{n}{n} \PY{o}{=}\PY{o}{=} \PY{l+m+mi}{2}\PY{o}{)} \PY{k}{return} \PY{l+m+mi}{1}\PY{o}{;}
            \PY{k}{return} \PY{n}{fib}\PY{o}{(}\PY{n}{n}\PY{o}{\PYZhy{}}\PY{l+m+mi}{2}\PY{o}{)} \PY{o}{+} \PY{n}{fib}\PY{o}{(}\PY{n}{n}\PY{o}{\PYZhy{}}\PY{l+m+mi}{1}\PY{o}{)}\PY{o}{;}
        \PY{o}{\PYZcb{}} \PY{c+c1}{// fib}
        
        \PY{k+kt}{int} \PY{n}{number} \PY{o}{=} \PY{l+m+mi}{5}\PY{o}{;}
        \PY{k}{for} \PY{o}{(}\PY{k+kt}{int} \PY{n}{i} \PY{o}{=} \PY{l+m+mi}{1}\PY{o}{;} \PY{n}{i} \PY{o}{\PYZlt{}}\PY{n}{number} \PY{o}{+}\PY{l+m+mi}{1} \PY{o}{;} \PY{n}{i}\PY{o}{+}\PY{o}{+}\PY{o}{)}
            \PY{n}{System}\PY{o}{.}\PY{n+na}{out}\PY{o}{.}\PY{n+na}{print}\PY{o}{(}\PY{n}{fib}\PY{o}{(}\PY{n}{i}\PY{o}{)}\PY{o}{+}\PY{l+s}{\PYZdq{} \PYZdq{}}\PY{o}{)}\PY{o}{;}
\end{Verbatim}


    \begin{Verbatim}[commandchars=\\\{\}]
{\color{incolor}In [{\color{incolor} }]:} \PY{c+c1}{//recursive Fibonacci numbers in order without the for loop}
        \PY{k+kt}{void} \PY{n+nf}{fib}\PY{o}{(}\PY{k+kt}{int} \PY{n}{f}\PY{o}{,}\PY{k+kt}{int} \PY{n}{s}\PY{o}{,}\PY{k+kt}{int} \PY{n}{t}\PY{o}{,}\PY{k+kt}{int} \PY{n}{limit}\PY{o}{)} \PY{o}{\PYZob{}}
            \PY{k}{if}\PY{o}{(}\PY{n}{limit}\PY{o}{\PYZlt{}}\PY{o}{=}\PY{l+m+mi}{0}\PY{o}{)} 
                \PY{k}{return} \PY{o}{;}
            \PY{n}{t} \PY{o}{=} \PY{n}{f}\PY{o}{+}\PY{n}{s}\PY{o}{;}
            \PY{n}{System}\PY{o}{.}\PY{n+na}{out}\PY{o}{.}\PY{n+na}{print}\PY{o}{(}\PY{n}{f}\PY{o}{+}\PY{l+s}{\PYZdq{} \PYZdq{}}\PY{o}{)}\PY{o}{;}
            \PY{n}{f}\PY{o}{=}\PY{n}{s}\PY{o}{;}
            \PY{n}{s}\PY{o}{=}\PY{n}{t}\PY{o}{;}
        
            \PY{n}{fib}\PY{o}{(}\PY{n}{f}\PY{o}{,} \PY{n}{s}\PY{o}{,} \PY{n}{t}\PY{o}{,} \PY{o}{\PYZhy{}}\PY{o}{\PYZhy{}}\PY{n}{limit}\PY{o}{)}\PY{o}{;}
        \PY{o}{\PYZcb{}}
\end{Verbatim}


    \begin{Verbatim}[commandchars=\\\{\}]
{\color{incolor}In [{\color{incolor} }]:} \PY{c+c1}{//recursive Fibonacci numbers in order without the for loop (continued)}
        \PY{k+kt}{int} \PY{n}{number} \PY{o}{=} \PY{l+m+mi}{5}\PY{o}{;}
        \PY{n}{fib}\PY{o}{(}\PY{l+m+mi}{1}\PY{o}{,} \PY{l+m+mi}{1}\PY{o}{,} \PY{l+m+mi}{0}\PY{o}{,} \PY{n}{number}\PY{o}{)}\PY{o}{;}
\end{Verbatim}


    Fibonacci: Iterative Approach

    \begin{Verbatim}[commandchars=\\\{\}]
{\color{incolor}In [{\color{incolor} }]:} \PY{k+kt}{int} \PY{n}{f}\PY{o}{=}\PY{l+m+mi}{1}\PY{o}{,}\PY{n}{s}\PY{o}{=}\PY{l+m+mi}{1}\PY{o}{,}\PY{n}{t} \PY{o}{=} \PY{l+m+mi}{0}\PY{o}{,}\PY{n}{count}\PY{o}{=}\PY{l+m+mi}{5}\PY{o}{;}    
            
        \PY{k}{for}\PY{o}{(}\PY{k+kt}{int} \PY{n}{i}\PY{o}{=}\PY{n}{count}\PY{o}{;}\PY{n}{i}\PY{o}{\PYZgt{}}\PY{l+m+mi}{0}\PY{o}{;}\PY{o}{\PYZhy{}}\PY{o}{\PYZhy{}}\PY{n}{i}\PY{o}{)}
        \PY{o}{\PYZob{}} 
            \PY{n}{t} \PY{o}{=} \PY{n}{f}\PY{o}{+}\PY{n}{s}\PY{o}{;}
            \PY{n}{System}\PY{o}{.}\PY{n+na}{out}\PY{o}{.}\PY{n+na}{print}\PY{o}{(}\PY{n}{f}\PY{o}{+}\PY{l+s}{\PYZdq{} \PYZdq{}}\PY{o}{)}\PY{o}{;}
            \PY{n}{f}\PY{o}{=}\PY{n}{s}\PY{o}{;}
            \PY{n}{s}\PY{o}{=}\PY{n}{t}\PY{o}{;} 
         \PY{o}{\PYZcb{}}
\end{Verbatim}


    Palindrome

    A palindrome is a word that is spelled the same forward and backward.

    For example: kayak, Anna, level, racecar are palindromes, but pipe is
not.

    Palindrome

    racecar

    aceca

    cec

    e

    Palindrome

    Find the longest Palindrome in a text

    \begin{Verbatim}[commandchars=\\\{\}]
{\color{incolor}In [{\color{incolor} }]:} \PY{k+kt}{boolean} \PY{n+nf}{isPal}\PY{o}{(}\PY{n}{String} \PY{n}{s}\PY{o}{)}
            \PY{o}{\PYZob{}} 
                \PY{k}{if}\PY{o}{(}\PY{n}{s}\PY{o}{.}\PY{n+na}{length}\PY{o}{(}\PY{o}{)} \PY{o}{=}\PY{o}{=} \PY{l+m+mi}{0} \PY{o}{|}\PY{o}{|} \PY{n}{s}\PY{o}{.}\PY{n+na}{length}\PY{o}{(}\PY{o}{)} \PY{o}{=}\PY{o}{=} \PY{l+m+mi}{1}\PY{o}{)}
                    \PY{k}{return} \PY{k+kc}{true}\PY{o}{;} 
                \PY{k}{if}\PY{o}{(}\PY{n}{s}\PY{o}{.}\PY{n+na}{charAt}\PY{o}{(}\PY{l+m+mi}{0}\PY{o}{)} \PY{o}{=}\PY{o}{=} \PY{n}{s}\PY{o}{.}\PY{n+na}{charAt}\PY{o}{(}\PY{n}{s}\PY{o}{.}\PY{n+na}{length}\PY{o}{(}\PY{o}{)}\PY{o}{\PYZhy{}}\PY{l+m+mi}{1}\PY{o}{)}\PY{o}{)}
                    \PY{k}{return} \PY{n}{isPal}\PY{o}{(}\PY{n}{s}\PY{o}{.}\PY{n+na}{substring}\PY{o}{(}\PY{l+m+mi}{1}\PY{o}{,} \PY{n}{s}\PY{o}{.}\PY{n+na}{length}\PY{o}{(}\PY{o}{)}\PY{o}{\PYZhy{}}\PY{l+m+mi}{1}\PY{o}{)}\PY{o}{)}\PY{o}{;}
                \PY{k}{return} \PY{k+kc}{false}\PY{o}{;}
            \PY{o}{\PYZcb{}}
\end{Verbatim}


    \begin{Verbatim}[commandchars=\\\{\}]
{\color{incolor}In [{\color{incolor} }]:} \PY{n}{String} \PY{n}{text} \PY{o}{=} \PY{l+s}{\PYZdq{}Wow, Hannah has bought a racecar from Laval.\PYZdq{}}\PY{o}{;}
                \PY{n}{String}\PY{o}{[}\PY{o}{]} \PY{n}{arr} \PY{o}{=} \PY{n}{text}\PY{o}{.}\PY{n+na}{split}\PY{o}{(}\PY{l+s}{\PYZdq{}\PYZbs{}\PYZbs{}.| |,\PYZdq{}}\PY{o}{)}\PY{o}{;}
                \PY{n}{String} \PY{n}{longest}\PY{o}{=}\PY{l+s}{\PYZdq{}\PYZdq{}}\PY{o}{;}
                \PY{k}{for} \PY{o}{(}\PY{k+kt}{int} \PY{n}{i}\PY{o}{=}\PY{l+m+mi}{0}\PY{o}{;} \PY{n}{i}\PY{o}{\PYZlt{}} \PY{n}{arr}\PY{o}{.}\PY{n+na}{length}\PY{o}{;} \PY{n}{i}\PY{o}{+}\PY{o}{+}\PY{o}{)}\PY{o}{\PYZob{}}
                    \PY{k}{if}\PY{o}{(}\PY{n}{isPal}\PY{o}{(}\PY{n}{arr}\PY{o}{[}\PY{n}{i}\PY{o}{]}\PY{o}{)}\PY{o}{)}\PY{o}{\PYZob{}}
                        \PY{k}{if}\PY{o}{(}\PY{n}{arr}\PY{o}{[}\PY{n}{i}\PY{o}{]}\PY{o}{.}\PY{n+na}{length}\PY{o}{(}\PY{o}{)}\PY{o}{\PYZgt{}}\PY{n}{longest}\PY{o}{.}\PY{n+na}{length}\PY{o}{(}\PY{o}{)}\PY{o}{)}
                            \PY{n}{longest} \PY{o}{=} \PY{n}{arr}\PY{o}{[}\PY{n}{i}\PY{o}{]}\PY{o}{;}
                    \PY{o}{\PYZcb{}}\PY{o}{\PYZcb{}}
                \PY{n}{System}\PY{o}{.}\PY{n+na}{out}\PY{o}{.}\PY{n+na}{println}\PY{o}{(}\PY{n}{longest} \PY{o}{+} \PY{l+s}{\PYZdq{} is the longest palindrome in your text!\PYZdq{}}\PY{o}{)}\PY{o}{;}
\end{Verbatim}


    Palindrome: Iterative Approach

    \begin{Verbatim}[commandchars=\\\{\}]
{\color{incolor}In [{\color{incolor}6}]:} \PY{k+kt}{boolean} \PY{n+nf}{checkPalindrome}\PY{o}{(}\PY{n}{String} \PY{n}{text}\PY{o}{)}\PY{o}{\PYZob{}} 
                \PY{n}{String} \PY{n}{reversed} \PY{o}{=} \PY{l+s}{\PYZdq{}\PYZdq{}}\PY{o}{;} 
                \PY{k+kt}{char}\PY{o}{[}\PY{o}{]} \PY{n}{wordLetters} \PY{o}{=} \PY{n}{text}\PY{o}{.}\PY{n+na}{toCharArray}\PY{o}{(}\PY{o}{)}\PY{o}{;}         
                \PY{k}{for}\PY{o}{(}\PY{k+kt}{int} \PY{n}{i} \PY{o}{=} \PY{n}{text}\PY{o}{.}\PY{n+na}{length}\PY{o}{(}\PY{o}{)} \PY{o}{\PYZhy{}}\PY{l+m+mi}{1}\PY{o}{;} \PY{n}{i}\PY{o}{\PYZgt{}}\PY{o}{=}\PY{l+m+mi}{0} \PY{o}{;} \PY{n}{i}\PY{o}{\PYZhy{}}\PY{o}{\PYZhy{}}\PY{o}{)} 
                    \PY{n}{reversed} \PY{o}{+}\PY{o}{=} \PY{n}{wordLetters}\PY{o}{[}\PY{n}{i}\PY{o}{]}\PY{o}{;} 
                \PY{k}{return} \PY{n}{text}\PY{o}{.}\PY{n+na}{equals}\PY{o}{(}\PY{n}{reversed}\PY{o}{)}\PY{o}{;} \PY{o}{\PYZcb{}}
\end{Verbatim}


    \begin{Verbatim}[commandchars=\\\{\}]
{\color{incolor}In [{\color{incolor}7}]:} \PY{n}{String} \PY{n}{text} \PY{o}{=} \PY{l+s}{\PYZdq{}Wow, Hannah has bought a racecar from Laval.\PYZdq{}}\PY{o}{;}
                \PY{n}{String}\PY{o}{[}\PY{o}{]} \PY{n}{arr} \PY{o}{=} \PY{n}{text}\PY{o}{.}\PY{n+na}{split}\PY{o}{(}\PY{l+s}{\PYZdq{}\PYZbs{}\PYZbs{}.| |,\PYZdq{}}\PY{o}{)}\PY{o}{;}
                \PY{n}{String} \PY{n}{longest}\PY{o}{=}\PY{l+s}{\PYZdq{}\PYZdq{}}\PY{o}{;}
                \PY{k}{for} \PY{o}{(}\PY{k+kt}{int} \PY{n}{i}\PY{o}{=}\PY{l+m+mi}{0}\PY{o}{;} \PY{n}{i}\PY{o}{\PYZlt{}} \PY{n}{arr}\PY{o}{.}\PY{n+na}{length}\PY{o}{;} \PY{n}{i}\PY{o}{+}\PY{o}{+}\PY{o}{)}\PY{o}{\PYZob{}}
                    \PY{k}{if}\PY{o}{(}\PY{n}{checkPalindrome}\PY{o}{(}\PY{n}{arr}\PY{o}{[}\PY{n}{i}\PY{o}{]}\PY{o}{)}\PY{o}{)}\PY{o}{\PYZob{}}
                        \PY{k}{if}\PY{o}{(}\PY{n}{arr}\PY{o}{[}\PY{n}{i}\PY{o}{]}\PY{o}{.}\PY{n+na}{length}\PY{o}{(}\PY{o}{)}\PY{o}{\PYZgt{}}\PY{n}{longest}\PY{o}{.}\PY{n+na}{length}\PY{o}{(}\PY{o}{)}\PY{o}{)}
                            \PY{n}{longest} \PY{o}{=} \PY{n}{arr}\PY{o}{[}\PY{n}{i}\PY{o}{]}\PY{o}{;} \PY{o}{\PYZcb{}}\PY{o}{\PYZcb{}}
                \PY{n}{System}\PY{o}{.}\PY{n+na}{out}\PY{o}{.}\PY{n+na}{println}\PY{o}{(}\PY{n}{longest} \PY{o}{+} \PY{l+s}{\PYZdq{} is the longest palindrome in your text!\PYZdq{}}\PY{o}{)}\PY{o}{;}
\end{Verbatim}


    \begin{Verbatim}[commandchars=\\\{\}]
racecar is the longest palindrome in your text!

    \end{Verbatim}

    Permutation of Letters in a Word

    \begin{itemize}
\tightlist
\item
  It is the number of ways we can change the order of the letters in a
  word.
\item
  Each different letter arrangement is called a permutation of the word.
\end{itemize}

    \begin{itemize}
\tightlist
\item
   Sample Word: STOP
\item
   All Permutations: \{STOP, STPO, SOTP, SOPT, SPTO, SPOT, TSOP, TSPO,
  TOSP, TOPS, TPSO, TPOS, OSTP, OSPT, OTSP, OTPS, OPST, OPTS, PSTO,
  PSOT, PTSO, PTOS, POST, POTS\}
\end{itemize}

    Password Checker

    

    Let's Think-Pair-Share!

    \begin{Verbatim}[commandchars=\\\{\}]
{\color{incolor}In [{\color{incolor} }]:} \PY{c+c1}{// Prints all the permutations of letters in a word}
        \PY{k+kt}{void} \PY{n+nf}{permutationPrint}\PY{o}{(}\PY{n}{String} \PY{n}{newWord}\PY{o}{,} \PY{n}{String} \PY{n}{word}\PY{o}{)} \PY{o}{\PYZob{}}
            \PY{k}{if} \PY{o}{(}\PY{n}{word}\PY{o}{.}\PY{n+na}{isEmpty}\PY{o}{(}\PY{o}{)}\PY{o}{)} \PY{o}{\PYZob{}}
                \PY{n}{System}\PY{o}{.}\PY{n+na}{out}\PY{o}{.}\PY{n+na}{print}\PY{o}{(}\PY{n}{newWord}\PY{o}{+}\PY{l+s}{\PYZdq{} \PYZdq{}}\PY{o}{)}\PY{o}{;}    
            \PY{o}{\PYZcb{}} 
            \PY{k}{else} \PY{o}{\PYZob{}}
                \PY{k}{for} \PY{o}{(}\PY{k+kt}{int} \PY{n}{i} \PY{o}{=} \PY{l+m+mi}{0}\PY{o}{;} \PY{n}{i} \PY{o}{\PYZlt{}} \PY{n}{word}\PY{o}{.}\PY{n+na}{length}\PY{o}{(}\PY{o}{)}\PY{o}{;} \PY{n}{i}\PY{o}{+}\PY{o}{+}\PY{o}{)} \PY{o}{\PYZob{}}
                    \PY{n}{permutationPrint}\PY{o}{(}\PY{n}{newWord} \PY{o}{+} \PY{n}{word}\PY{o}{.}\PY{n+na}{charAt}\PY{o}{(}\PY{n}{i}\PY{o}{)}\PY{o}{,} \PY{n}{word}\PY{o}{.}\PY{n+na}{substring}\PY{o}{(}\PY{l+m+mi}{0}\PY{o}{,} \PY{n}{i}\PY{o}{)} \PY{o}{+} \PY{n}{word}\PY{o}{.}\PY{n+na}{substring}\PY{o}{(}\PY{n}{i} \PY{o}{+} \PY{l+m+mi}{1}\PY{o}{)}\PY{o}{)}\PY{o}{;}
                \PY{o}{\PYZcb{}}
            \PY{o}{\PYZcb{}}
        \PY{o}{\PYZcb{}}
\end{Verbatim}


    \begin{Verbatim}[commandchars=\\\{\}]
{\color{incolor}In [{\color{incolor} }]:} \PY{n}{permutationPrint}\PY{o}{(}\PY{l+s}{\PYZdq{}\PYZdq{}}\PY{o}{,} \PY{l+s}{\PYZdq{}STOP\PYZdq{}}\PY{o}{)}\PY{o}{;}
\end{Verbatim}


    \begin{Verbatim}[commandchars=\\\{\}]
{\color{incolor}In [{\color{incolor}12}]:} \PY{c+c1}{// Creates all the permutations of letters in a word and store them in a list.}
         
         \PY{k+kt}{void} \PY{n+nf}{permutationList}\PY{o}{(}\PY{n}{String} \PY{n}{newWord}\PY{o}{,} \PY{n}{String} \PY{n}{word}\PY{o}{,} \PY{n}{List} \PY{n}{words}\PY{o}{)} \PY{o}{\PYZob{}}
         
             \PY{k}{if} \PY{o}{(}\PY{n}{word}\PY{o}{.}\PY{n+na}{isEmpty}\PY{o}{(}\PY{o}{)}\PY{o}{)} 
                 \PY{n}{words}\PY{o}{.}\PY{n+na}{add}\PY{o}{(}\PY{n}{newWord}\PY{o}{)}\PY{o}{;}     
             \PY{k}{else} \PY{o}{\PYZob{}}
                 \PY{k}{for} \PY{o}{(}\PY{k+kt}{int} \PY{n}{i} \PY{o}{=} \PY{l+m+mi}{0}\PY{o}{;} \PY{n}{i} \PY{o}{\PYZlt{}} \PY{n}{word}\PY{o}{.}\PY{n+na}{length}\PY{o}{(}\PY{o}{)}\PY{o}{;} \PY{n}{i}\PY{o}{+}\PY{o}{+}\PY{o}{)} \PY{o}{\PYZob{}}
                     \PY{n}{permutationList}\PY{o}{(}\PY{n}{newWord} \PY{o}{+} \PY{n}{word}\PY{o}{.}\PY{n+na}{charAt}\PY{o}{(}\PY{n}{i}\PY{o}{)}\PY{o}{,} \PY{n}{word}\PY{o}{.}\PY{n+na}{substring}\PY{o}{(}\PY{l+m+mi}{0}\PY{o}{,} \PY{n}{i}\PY{o}{)} \PY{o}{+} \PY{n}{word}\PY{o}{.}\PY{n+na}{substring}\PY{o}{(}\PY{n}{i} \PY{o}{+} \PY{l+m+mi}{1}\PY{o}{)}\PY{o}{,}\PY{n}{words}\PY{o}{)}\PY{o}{;}\PY{o}{\PYZcb{}}\PY{o}{\PYZcb{}}\PY{o}{\PYZcb{}}
\end{Verbatim}


    \begin{Verbatim}[commandchars=\\\{\}]
{\color{incolor}In [{\color{incolor} }]:} \PY{n}{String} \PY{n}{oldPass}\PY{o}{=}\PY{l+s}{\PYZdq{}joe75\PYZdq{}}\PY{o}{;}
        \PY{n}{String} \PY{n}{newPass}\PY{o}{=}\PY{l+s}{\PYZdq{}\PYZdq{}}\PY{o}{;}
        \PY{n}{List} \PY{n}{list} \PY{o}{=} \PY{k}{new} \PY{n}{ArrayList}\PY{o}{\PYZlt{}}\PY{n}{String}\PY{o}{\PYZgt{}}\PY{o}{(}\PY{o}{)}\PY{o}{;}
        \PY{n}{Scanner} \PY{n}{scanner} \PY{o}{=} \PY{k}{new} \PY{n}{Scanner}\PY{o}{(}\PY{n}{System}\PY{o}{.}\PY{n+na}{in}\PY{o}{)}\PY{o}{;}
        \PY{n}{System}\PY{o}{.}\PY{n+na}{out}\PY{o}{.}\PY{n+na}{println}\PY{o}{(}\PY{l+s}{\PYZdq{}Please insert your new password:\PYZdq{}}\PY{o}{)}\PY{o}{;}
        \PY{k}{do}\PY{o}{\PYZob{}}
            \PY{n}{newPass} \PY{o}{=} \PY{n}{scanner}\PY{o}{.}\PY{n+na}{next}\PY{o}{(}\PY{o}{)}\PY{o}{;} 
            \PY{n}{list}\PY{o}{.}\PY{n+na}{clear}\PY{o}{(}\PY{o}{)}\PY{o}{;}
            \PY{n}{permutationList}\PY{o}{(}\PY{l+s}{\PYZdq{}\PYZdq{}}\PY{o}{,} \PY{n}{newPass}\PY{o}{,}\PY{n}{list}\PY{o}{)}\PY{o}{;}
            \PY{c+c1}{//System.out.println(list);}
            \PY{k}{if}\PY{o}{(}\PY{o}{!}\PY{n}{list}\PY{o}{.}\PY{n+na}{contains}\PY{o}{(}\PY{n}{oldPass}\PY{o}{)}\PY{o}{)}
                \PY{k}{break}\PY{o}{;}
            \PY{k}{else}\PY{o}{\PYZob{}}
                \PY{n}{System}\PY{o}{.}\PY{n+na}{err}\PY{o}{.}\PY{n+na}{println}\PY{o}{(}\PY{l+s}{\PYZdq{}Your new password is a permutation of your current one!\PYZdq{}}\PY{o}{)}\PY{o}{;}
                \PY{n}{System}\PY{o}{.}\PY{n+na}{out}\PY{o}{.}\PY{n+na}{println}\PY{o}{(}\PY{l+s}{\PYZdq{}Please insert your new password:\PYZdq{}}\PY{o}{)}\PY{o}{;}
                \PY{o}{\PYZcb{}}       
        \PY{o}{\PYZcb{}}\PY{k}{while}\PY{o}{(}\PY{k+kc}{true}\PY{o}{)}\PY{o}{;}
        \PY{n}{System}\PY{o}{.}\PY{n+na}{out}\PY{o}{.}\PY{n+na}{println}\PY{o}{(}\PY{l+s}{\PYZdq{}\PYZbs{}nYou succesfully changed the password!\PYZdq{}}\PY{o}{)}\PY{o}{;}
\end{Verbatim}


    \begin{Verbatim}[commandchars=\\\{\}]
Please insert your new password:
joe57

    \end{Verbatim}

    \begin{Verbatim}[commandchars=\\\{\}]
Your new password is a permutation of your current one!

    \end{Verbatim}

    \begin{Verbatim}[commandchars=\\\{\}]
Please insert your new password:

    \end{Verbatim}

    Permutation of Letters in a Word: Iterative Approach

    \begin{Verbatim}[commandchars=\\\{\}]
{\color{incolor}In [{\color{incolor} }]:} \PY{n}{String} \PY{n}{word} \PY{o}{=} \PY{l+s}{\PYZdq{}SUN\PYZdq{}}\PY{o}{;}
        
        \PY{n}{List}\PY{o}{\PYZlt{}}\PY{n}{String}\PY{o}{\PYZgt{}} \PY{n}{newWord} \PY{o}{=} \PY{k}{new} \PY{n}{ArrayList}\PY{o}{\PYZlt{}}\PY{o}{\PYZgt{}}\PY{o}{(}\PY{o}{)}\PY{o}{;}
        \PY{n}{newWord}\PY{o}{.}\PY{n+na}{add}\PY{o}{(}\PY{n}{String}\PY{o}{.}\PY{n+na}{valueOf}\PY{o}{(}\PY{n}{word}\PY{o}{.}\PY{n+na}{charAt}\PY{o}{(}\PY{l+m+mi}{0}\PY{o}{)}\PY{o}{)}\PY{o}{)}\PY{o}{;}
        
        \PY{k}{for} \PY{o}{(}\PY{k+kt}{int} \PY{n}{i} \PY{o}{=} \PY{l+m+mi}{1}\PY{o}{;} \PY{n}{i} \PY{o}{\PYZlt{}} \PY{n}{word}\PY{o}{.}\PY{n+na}{length}\PY{o}{(}\PY{o}{)}\PY{o}{;} \PY{n}{i}\PY{o}{+}\PY{o}{+}\PY{o}{)}
        \PY{o}{\PYZob{}}
            \PY{k}{for} \PY{o}{(}\PY{k+kt}{int} \PY{n}{j} \PY{o}{=} \PY{n}{newWord}\PY{o}{.}\PY{n+na}{size}\PY{o}{(}\PY{o}{)} \PY{o}{\PYZhy{}} \PY{l+m+mi}{1}\PY{o}{;} \PY{n}{j} \PY{o}{\PYZgt{}}\PY{o}{=} \PY{l+m+mi}{0} \PY{o}{;} \PY{n}{j}\PY{o}{\PYZhy{}}\PY{o}{\PYZhy{}}\PY{o}{)}
            \PY{o}{\PYZob{}}
                \PY{n}{String} \PY{n}{str} \PY{o}{=} \PY{n}{newWord}\PY{o}{.}\PY{n+na}{remove}\PY{o}{(}\PY{n}{j}\PY{o}{)}\PY{o}{;}
        
                \PY{k}{for} \PY{o}{(}\PY{k+kt}{int} \PY{n}{k} \PY{o}{=} \PY{l+m+mi}{0}\PY{o}{;} \PY{n}{k} \PY{o}{\PYZlt{}}\PY{o}{=} \PY{n}{str}\PY{o}{.}\PY{n+na}{length}\PY{o}{(}\PY{o}{)}\PY{o}{;} \PY{n}{k}\PY{o}{+}\PY{o}{+}\PY{o}{)}
                \PY{o}{\PYZob{}}            
                    \PY{n}{newWord}\PY{o}{.}\PY{n+na}{add}\PY{o}{(}\PY{n}{str}\PY{o}{.}\PY{n+na}{substring}\PY{o}{(}\PY{l+m+mi}{0}\PY{o}{,} \PY{n}{k}\PY{o}{)} \PY{o}{+} \PY{n}{word}\PY{o}{.}\PY{n+na}{charAt}\PY{o}{(}\PY{n}{i}\PY{o}{)} \PY{o}{+} \PY{n}{str}\PY{o}{.}\PY{n+na}{substring}\PY{o}{(}\PY{n}{k}\PY{o}{)}\PY{o}{)}\PY{o}{;}
                \PY{o}{\PYZcb{}}
            \PY{o}{\PYZcb{}}
        \PY{o}{\PYZcb{}}
        \PY{n}{System}\PY{o}{.}\PY{n+na}{out}\PY{o}{.}\PY{n+na}{println}\PY{o}{(}\PY{n}{newWord}\PY{o}{)}\PY{o}{;}
\end{Verbatim}


    File Systems

    

    List All the Files in a Folder Structure

    \begin{Verbatim}[commandchars=\\\{\}]
{\color{incolor}In [{\color{incolor} }]:} \PY{k+kn}{import} \PY{n+nn}{java.io.File}\PY{o}{;}
        \PY{k+kt}{void} \PY{n+nf}{dirTree}\PY{o}{(}\PY{n}{File} \PY{n}{dir}\PY{o}{)} \PY{o}{\PYZob{}}
              \PY{n}{File}\PY{o}{[}\PY{o}{]} \PY{n}{subdirs}\PY{o}{=}\PY{n}{dir}\PY{o}{.}\PY{n+na}{listFiles}\PY{o}{(}\PY{o}{)}\PY{o}{;}
              \PY{k}{for}\PY{o}{(}\PY{n}{File} \PY{n}{subdir}\PY{o}{:} \PY{n}{subdirs}\PY{o}{)} \PY{o}{\PYZob{}}
                 \PY{k}{if} \PY{o}{(}\PY{n}{subdir}\PY{o}{.}\PY{n+na}{isDirectory}\PY{o}{(}\PY{o}{)}\PY{o}{)} 
                    \PY{n}{dirTree}\PY{o}{(}\PY{n}{subdir}\PY{o}{)}\PY{o}{;}
                 \PY{k}{else} 
                    \PY{n}{System}\PY{o}{.}\PY{n+na}{out}\PY{o}{.}\PY{n+na}{println}\PY{o}{(}\PY{n}{subdir}\PY{o}{.}\PY{n+na}{getAbsolutePath}\PY{o}{(}\PY{o}{)}\PY{o}{)}\PY{o}{;}
                \PY{o}{\PYZcb{}}
           \PY{o}{\PYZcb{}}
\end{Verbatim}


    \begin{Verbatim}[commandchars=\\\{\}]
{\color{incolor}In [{\color{incolor} }]:} \PY{n}{dirTree}\PY{o}{(}\PY{k}{new} \PY{n}{File}\PY{o}{(}\PY{l+s}{\PYZdq{}/Users/navid/Desktop/Spring2019/SystemsReadingGroup\PYZdq{}}\PY{o}{)}\PY{o}{)}\PY{o}{;}
\end{Verbatim}


    Review: Recursion Steps

    \begin{itemize}
\item
  Step 1:~Determine what the function is.
\item
  Step 2: Determine the base case(s) and the solution.
\item
  Step 3: Determine what should be included in the Recursive Call.
\item
  Step 4: Trace and test for different inputs.
\end{itemize}

    


    % Add a bibliography block to the postdoc
    
    
    
    \end{document}
